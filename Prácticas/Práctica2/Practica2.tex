\documentclass[answers]{exam}

\usepackage[spanish]{babel}
\usepackage{amsmath}
\usepackage[T1]{fontenc}
\usepackage{graphicx}
\usepackage{karnaugh-map}
\usepackage{lmodern}
\usepackage{multicol}
\usepackage{wrapfig}

\extrawidth{1.04cm}
\extraheadheight[2.5cm]{-5mm}
\renewcommand{\familydefault}{\sfdefault}
\graphicspath{{../../Imagenes/}}

\newcommand{\materia}{Organización y \\ Arquitectura de Computadoras}
\newcommand{\tarea}{Práctica 2}
\newcommand{\titulo}{Circuitos Combinacionales}
\newcommand{\fecha}{28 de Octubre de 2021}

\firstpageheader{
  \setlength{\intextsep}{2.2em}
  \begin{wrapfigure}{l}{3.3cm}
    \centering
    \includegraphics[scale=0.08]{fc}
  \end{wrapfigure}
  \hfill\break{} \\[3.5mm]
  \LARGE\textbf{\materia} \\
  \large\textbf{Facultad de Ciencias, UNAM} \\[4pt]
  \textbf{José Ethan Ortega González: }316088327 \\
  \textbf{Etzael Iván Sosa Hedding: }316259305 \\
}{}{
  \setlength{\intextsep}{-9.8em}
  \begin{wrapfigure}{l}{3.3cm}
    \centering
    \includegraphics[scale=0.13]{unam}
  \end{wrapfigure}
}

\renewcommand{\solutiontitle}{\noindent\textbf{Solución:}\par\noindent}
\renewcommand{\thequestion}{\textbf{\arabic{question}}}
\runningheadrule{}
\runningfootrule{}
\firstpagefootrule{}
\runningheader{\materia}{\tarea: \titulo}{\fecha}
\footer{}{Página \thepage\ de \numpages}{}

\begin{document}
\section{Procedimiento}
\begin{questions}
  \question{Construye las 3 compuertas básicas: Not, Or y And. Para corroborar
    la funcionalidad de sus compuertas, deberán construir circuitos para simular
    las siguientes funciones. Sólo puedes hacer uso de fuentes de alimentación
    power y ground, transistores, resistencias y pines de entrada y salida.
    \begin{enumerate}
      \item $A \leftrightarrow B$
      \item $A \oplus B$
    \end{enumerate}}
  \begin{solution}
    \begin{itemize}
      \item \textbf{NOT:} Para este circuito pensamos en dejar pasar la
            corriente de la fuente en un caso y bloquearla en el otro,
            auxiliándonos de un elemento tierra para indicar el 0. Utilizamos un
            transistor de tipo P que deja pasar la corriente cuando recibe un 0
            y uno de tipo N que está abierto cuando recibe un 1. Es claro que
            necesitamos uno de cada tipo para que siempre haya uno abierto y
            deje pasar la información deseada.
      \item \textbf{AND:} Como necesitamos que ambas entradas sean 1, ponemos 2
            transistores de tipo N para que dejen pasar la corriente de la
            fuente de poder al ser 1 ambos pues si no fueran iguales no pasaría
            la corriente ni tampoco pueden ser ambos de tipo P porque esas se
            cierran al ser 1. Sin embargo, también es necesario poner una
            resistencia para que por defecto tengamos un 0, ya que si ambas
            entradas son 0 no pasa nada y estaría indeterminada la salida.
      \item \textbf{OR:} Aquí necesitamos 2 transistores de tipo N para que
            dejen pasar la corriente cuando sean 1 pero necesitamos un
            transistor con 0 para que por defecto (cuando ambas entradas sean 0)
            se muestre un 0. Aquí con que una entrada sea 1 se permite pasar la
            corriente, en AND se tenía que dejar pasar la corriente por un
            transistor y luego por otro, aquí no ocurre eso, son independientes.
      \item \textbf{XOR:} Se evalúa a 1 cuando las entradas son distintas, por
            lo que utilizamos la definición que es la (negación de A y B) y (A o
            B) por lo que utilizamos los anteriores circuitos.
      \item \textbf{Equivalencia:} Como la tabla de verdad de XOR y la
            equivalencia son opuestas, basta con negar XOR.\@
    \end{itemize}
  \end{solution}

  \question{Construye un circuito que resuelva las situaciones que se piden.
    Debes hacer uso de las compuertas que construiste en el ejercicio anterior.
    \begin{enumerate}
      \item Indicar si un número $n$ es primo, con $n \in {0, \ldots, 15}$.
      \item Simular la siguiente función y reduce mediante mintérminos y
            maxtérminos. Agrega ambos circuitos.
            \begin{gather*}
              \begin{array}{|c|c|c|c|}
                \hline
                E3 & E2 & E1 & F \\
                \hline
                0 & 0 & 0 & 1 \\
                0 & 0 & 1 & 1 \\
                0 & 1 & 0 & 0 \\
                0 & 1 & 1 & 1 \\
                1 & 0 & 0 & 0 \\
                1 & 0 & 1 & 1 \\
                1 & 1 & 0 & 0 \\
                1 & 1 & 1 & 0 \\
                \hline
              \end{array}
            \end{gather*}
    \end{enumerate}}
  \begin{solution}
    \begin{enumerate}
      \item Primero identificamos los números primos del $0$ al $15$ e hicimos
            una tabla de verdad de $4$ variables para poder representar los
            números. Para cada número primo el resultado de la función $F$ será
            evaluada a $1$, en otro caso se evalúa a $0$. La tabla se muestra a
            continuación
            \begin{gather*}
              \begin{array}{|c|c|c|c|c|}
                \hline
                x_{0} & x_{1} & x_{2} & x_{3} & F \\
                \hline
                0     & 0     & 0     & 0     & 0 \\
                0     & 0     & 0     & 1     & 0 \\
                0     & 0     & 1     & 0     & 1 \\
                0     & 0     & 1     & 1     & 1 \\
                0     & 1     & 0     & 0     & 0 \\
                0     & 1     & 0     & 1     & 1 \\
                0     & 1     & 1     & 0     & 0 \\
                0     & 1     & 1     & 1     & 1 \\
                1     & 0     & 0     & 0     & 0 \\
                1     & 0     & 0     & 1     & 0 \\
                1     & 0     & 1     & 0     & 0 \\
                1     & 0     & 1     & 1     & 1 \\
                1     & 1     & 0     & 0     & 0 \\
                1     & 1     & 0     & 1     & 1 \\
                1     & 1     & 1     & 0     & 0 \\
                1     & 1     & 1     & 1     & 0 \\
                \hline
              \end{array}
            \end{gather*}
            Apoyándonos en la tabla de verdad, obtuvimos la siguiente función:
            \begin{gather*}
              F(x_{1}, x_{2}, x_{3}, x_{4}) =%
              \overline{x_{1}}\;\overline{x_{2}}x_{3}\overline{x_{4}} +%
              \overline{x_{1}}\;\overline{x_{2}}x_{3}x_{4} +%
              \overline{x_{1}}x_{2}\overline{x_{3}}x_{4} +%
              \overline{x_{1}}x_{2}x_{3}x_{4} +%
              x_{1}\overline{x_{2}}x_{3}x_{4} +%
              x_{1}x_{2}\;\overline{x_{3}}x_{4}
            \end{gather*}
            Podemos utilizar un mapa de Karnaugh para simplificar la función:
            \begin{center}
              \begin{karnaugh-map}[4][4][1][$x_{0}x_{1}$][$x_{2}x_{3}$]
                \minterms{2,3,5,7,11,13}
                \autoterms[0]
                \implicant{5}{7}
                \implicant{5}{13}
                \implicant{3}{2}
                \implicantedge{3}{3}{11}{11}
              \end{karnaugh-map}
            \end{center}
            Del mapa de Karnaugh, obtenemos la siguiente función:
            \begin{gather*}
              F(x_{1}, x_{2}, x_{3}, x_{4}) =%
              x_{1}\overline{x_{2}}x_{3} + \overline{x_{0}}x_{1}x_{3} +%
              \overline{x_{0}}\;\overline{x_{1}}x_{2} + \overline{x_{1}}x_{2}x_{3}
            \end{gather*}
            Basta con representar la función $F$ en un circuito para resolver el
            ejercicio.
      \item Como debemos de resolver el ejercicio con mintérminos y maxtérminos,
            explicamos el procedimiento para ambos:
            \begin{itemize}
              \item Para los \textbf{mintérminos} tenemos que fijarnos en las filas de
                    la tabla que se evalúan a 1. Cuando una entrada es 0 la
                    denotamos con la negación y cuando es 1 la dejamos como
                    está. Cambiaremos los nombres de las entradas por
                    conveniencia. Sea $E3 = A, E2 = B$ y $E1 = C$. Entonces tenemos
                    lo siguiente:
                    \begin{align*}
                      \overline{A}\;\;\overline{B}\;\;\overline{C} +%
                      \overline{A}\;\;\overline{B}C + \overline{A}BC + A\overline{B}C
                      &= \overline{A}(\overline{B}\;\overline{C} + \overline{B}C + BC)%
                        + A\overline{B}C \tag{por distributividad} \\
                      &= \overline{A}(\overline{B}(\overline{C} + C) + BC)%
                        + A\overline{B}C \tag{por distributividad} \\
                      &= \overline{A}(\overline{B} (1) + BC)%
                        + A\overline{B}C \tag{por complemento} \\
                      &= \overline{A}(\overline{B} + BC)%
                        + A\overline{B}C \tag{por identidad} \\
                      &= \overline{A}\;\overline{B} + \overline{A}BC%
                        + A\overline{B}C \tag{por distributividad}
                    \end{align*}
              \item Para los maxtérminos tenemos que fijarnos en las filas de
                    la tabla que se evalúan a 0. Cuando una entrada es 1 la
                    denotamos con la negación y cuando es 0 la dejamos como
                    está. Entonces tenemos lo siguiente:
                    \begin{gather*}
                      (A + \overline{B} + C)(\overline{A} + B + C)(\overline{A} +%
                      \overline{B} + C)(\overline{A} + \overline{B} + \overline{C})
                    \end{gather*}
                    Sea D $= \overline{B} + C$. Entonces tenemos:
                    \begin{gather*}
                      (A + D)(\overline{A} + B + C)(\overline{A} + D)(\overline{A} +%
                      \overline{B} + \overline{C})
                    \end{gather*}
                    Por conmutatividad tenemos:
                    \begin{gather*}
                      (A + D)(\overline{A} + D)(\overline{A} + B + C)(\overline{A} +%
                      \overline{B} + \overline{C})
                    \end{gather*}
                    Por distributividad:
                    \begin{gather*}
                      (\overline{A}(A + D) + D(A + D))(\overline{A} + B + C)(\overline{A}%
                      + \overline{B} + \overline{C})
                    \end{gather*}
                    Por distributividad:
                    \begin{gather*}
                      (\overline{A}A + \overline{A}D + DA + DD)(\overline{A} + B +%
                      C)(\overline{A} + \overline{B} + \overline{C})
                    \end{gather*}
                    Por complemento y por idempotencia:
                    \begin{gather*}
                      (0 + \overline{A}D + DA + D)(\overline{A} + B + C)(\overline{A} +%
                      \overline{B} + \overline{C})
                    \end{gather*}
                    Por identidad y por conmutatividad:
                    \begin{gather*}
                      (D\overline{A} + DA + D)(\overline{A} + B + C)(\overline{A} +%
                      \overline{B} + \overline{C})
                    \end{gather*}
                    Por distributividad y por complemento:
                    \begin{gather*}
                      (D(\overline{A} + A) + D)(\overline{A} + B + C)(\overline{A} +%
                      \overline{B} + \overline{C}) = (D(1) + D)(\overline{A} + B +%
                      C)(\overline{A} + \overline{B} + \overline{C})
                    \end{gather*}
                    Por identidad, por idempotencia y sustituyendo D:
                    \begin{gather*}
                      (D)(\overline{A} + B + C)(\overline{A} + \overline{B} +%
                      \overline{C}) = (\overline{B} + C)(\overline{A} + B +%
                      C)(\overline{A} + \overline{B} + \overline{C})
                    \end{gather*}
            \end{itemize}
    \end{enumerate}
  \end{solution}

  \question{En una planta de manejo de residuos tóxicos cuentan con la mejor
    tecnología para tratar desechos peligrosos y mantener a sus trabajadores
    seguros. Una parte fundamental de su sistema de protección consta de tres
    filtros que mantienen la toxicidad del área prácticamente nula. Sin embargo
    estos filtros pueden fallar, lo que volvería nocivo permanecer en la planta.
    Con la finalidad de monitorear el estado de los filtros, estos tienen un
    sensor que indica si es que el filtro esta fallando o no. Si uno de los
    filtros falla es posible trabajar con normalidad pero es necesario notificar
    al servicio técnico para que lo reparen a la brevedad. En caso de que fallen
    dos, la planta puede seguir trabajando pero los empleados se deben retirar
    mas temprano. Finalmente, si es que fallan los 3, sera necesario activar el
    protocolo de alerta y evacuar inmediatamente la planta. Se necesita un
    mecanismo que indique la cantidad de filtros que fallan para que puedan
    tomar las acciones pertinentes según sea el caso.}
  \begin{solution}
    Para poder representar el problema de la planta, lo que hicimos fue crear un
    circuito con tres entrada y tres salidas; si ninguna de las salidas está
    encendida, entonces ningún filtro está averiado y todo puede transcurrir con
    normalidad; si la primer salida está encendida, entonces exactamente un
    filtro de los tres que hay está averiado, por lo que se puede trabajar con
    normalidad, pero se debe de notificar al servicio técnico; si la segunda
    salida está encendida, entonces exactamente dos filtros están averiados, por
    lo que se puede trabajar con normalidad pero los trabajadores se deben de
    retirar más temprano; si la tercer salida está encendida, entonces los tres
    filtros están averiados, por lo que se activa el protocolo de alerta se
    activa y todos deben de evacuar.

    Para implementar esto, hicimos una tabla de verdad de tres variables y tres
    funciones de conmutación $F_{1}, F_{2}, F_{3}$ donde cada función
    corresponde a cada salida y el número de radares averiados; es decir,
    $F_{1}$ verifica si un radar está averiado y se conecta a la primer salida,
    $F_{2}$ verifica si dos radares estén averiados y se conecta a la segunda
    salida y $F_{3}$ verifica si tres radares están averiados y se conecta a la
    tercer salida. La tabla de verdad y las funciones de conmutación se muestran
    a continuación:

    \begin{minipage}{0.5\textwidth}
      \begin{gather*}
        \text{\textbf{Tabla de verdad:}} \\
        \begin{array}{|c|c|c|c|c|c|}
          \hline
          x & y & z & F_{1} & F_{2} & F_{3} \\
          \hline
          0 & 0 & 0 & 0 & 0 & 0 \\
          0 & 0 & 1 & 1 & 0 & 0 \\
          0 & 1 & 0 & 1 & 0 & 0 \\
          0 & 1 & 1 & 0 & 1 & 0 \\
          1 & 0 & 0 & 1 & 0 & 0 \\
          1 & 0 & 1 & 0 & 1 & 0 \\
          1 & 1 & 0 & 0 & 1 & 0 \\
          1 & 1 & 1 & 0 & 0 & 1 \\
          \hline
        \end{array}
      \end{gather*}
    \end{minipage}
    \begin{minipage}{0.3\textwidth}
      \textbf{Funciones de conmutación:}
      \begin{align*}
        F_{1}(x, y, z) &= \overline{x}\;\overline{y}z + \overline{x}b\overline{c}%
                         + a\overline{b}\;\overline{c} \\
        F_{2}(x, y, z) &= \overline{x}yz + x\overline{b}c + ab\overline{c} \\
        F_{3}(x, y, z) &= abc
      \end{align*}
    \end{minipage}
  \end{solution}
\end{questions}
\section{Preguntas}
\begin{questions}
  \question{¿Cuál es el procedimiento a seguir para desarrollar un circuito que
    resuelva un problema que involucre lógica combinacional?}
  \begin{solution}
    \begin{itemize}
      \item \textbf{Paso 1:} Entender y plantear el problema en términos de
            lógica combinacional, es decir, poner el problema en términos de 0's
            y 1's; saber cuándo cierta entrada nos debe dar una salida en
            particular.
      \item \textbf{Paso 2:} Aunque es opcional, es buena idea hacer una tabla
            de verdad para visualizar todas las posibles combinaciones de la
            entrada y las posibles salidas.
      \item \textbf{Paso 3:} Convertir los resultados de la tabla a una función
            de conmutación.
      \item \textbf{Paso 4:} Aunque podría ser parte del paso 3, es importante
            minimizar con algún método, ya sea mediante minitérminos,
            maxitérminos o mapas de Karnaugh, según convenga.
      \item \textbf{Paso 5:} Implementar los circuitos necesarios para el
            problema.
      \item \textbf{Paso 6:} Asegurarse de que funciona para cada
            combinación de la entrada.
    \end{itemize}
  \end{solution}

  \question{Si una función de conmutación se evalúa a más ceros que unos ¿es
    conveniente usar mintérminos o maxtérminos? ¿En el caso que se evalúe a más
    unos que ceros?}
  \begin{solution}
    Si una función se evalúa a más ceros que unos conviene utilizar minitérminos
    porque así nos tenemos que fijar únicamente en los resultados que se evalúan
    a unos (que son menos). En el caso contrario podemos utilizar maxitérminos
    que se enfocan en los resultados evaluados a cero.
  \end{solution}

  \question{Analizando el trabajo realizado, ¿cuáles son los inconvenientes de
    desarrollar los circuitos de forma manual?}
  \begin{solution}
    \begin{itemize}
      \item Si el problema no se entiende bien es muy fácil diseñar algo erróneo
            o hecho para alguna otra situación diferente de la que nos
            solicitan.
      \item Mientras más entradas o combinaciones tengamos más complejo se
            vuelve el diseño y por lo tanto más difícil se vuelve encontrar
            alguna falla o incluso se puede hacer confusa su implementación.
    \end{itemize}
  \end{solution}
\end{questions}
\end{document}
