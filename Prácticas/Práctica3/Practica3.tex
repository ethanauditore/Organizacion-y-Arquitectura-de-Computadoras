\documentclass[answers]{exam}

\usepackage[spanish]{babel}
\usepackage[T1]{fontenc}
\usepackage{lmodern}
\usepackage{graphicx}
\usepackage{wrapfig}

\extrawidth{1.04cm}
\extraheadheight[2.5cm]{-5mm}
\renewcommand{\familydefault}{\sfdefault}
\graphicspath{{../../Imagenes/}}

\newcommand{\materia}{Organización y \\ Arquitectura de Computadoras}
\newcommand{\tarea}{Práctica 3}
\newcommand{\titulo}{Unidad Aritmético Lógica}
\newcommand{\fecha}{28 de Octubre de 2021}

\firstpageheader{
  \setlength{\intextsep}{2.2em}
  \begin{wrapfigure}{l}{3.3cm}
    \centering
    \includegraphics[scale=0.08]{fc}
  \end{wrapfigure}
  \hfill\break{} \\[3.5mm]
  \LARGE\textbf{\materia} \\
  \large\textbf{Facultad de Ciencias, UNAM} \\[4pt]
  \textbf{José Ethan Ortega González: }316088327 \\
  \textbf{Etzael Iván Sosa Hedding: }316259305 \\
}{}{
  \setlength{\intextsep}{-9.8em}
  \begin{wrapfigure}{l}{3.3cm}
    \centering
    \includegraphics[scale=0.13]{unam}
  \end{wrapfigure}
}

\renewcommand{\solutiontitle}{\noindent\textbf{Solución:}\par\noindent}
\renewcommand{\thequestion}{\textbf{\arabic{question}}}
\runningheadrule{}
\runningfootrule{}
\firstpagefootrule{}
\runningheader{\materia}{\tarea: \titulo}{\fecha}
\footer{}{Página \thepage\ de \numpages}{}

\begin{document}
\begin{questions}
  \question{¿Qué operaciones aritméticas y lógicas son básicas para un
    procesador?}
  \begin{solution}
    Un procesador necesita únicamente 3 operaciones lógicas (AND, OR, NOT)
    pues con estas basta para realizar cualquier otra operación lógica. Por otro
    lado, con una suma y un comparador (saber si es menor, mayor o igual un par
    de números) es suficiente para las operaciones aritméticas pues la resta
    puede verse como una suma y otras operaciones aritméticas se pueden ver como
    sumas (multiplicación, división, etc.).

    Sin embargo, otras fuentes consideran las siguientes como las operaciones
    básicas de un procesador: AND, OR, NOT, suma, resta, multiplicación,
    división e inverso.
  \end{solution}

  \question{El diseño utilizado para realizar la adición resulta ser
    ineficiente, ¿por qué?

    ¿Qué tipo de sumador resulta ser más eficiente?}
  \begin{solution}
    En primer lugar, tener un semisumador puede resultar más complejo,
    dándonos un nuevo nivel de abstracción al necesitar un semisumador para
    realizar un sumador general. Es claro también que para un ALU de 8 bits
    necesitamos 8 sumadores previamente realizados correctamente y para ALUs de
    más bits el número de sumadores aumenta, haciendo confuso y ``revuelta'' la
    solución. Debemos mencionar también el ``Overflow'' o desbordamiento, que es
    el acarreo que presenta el sumador si la suma de nuestras dos entradas
    necesita más bits de los que tenemos para ser representada. Esto es
    importante porque de no considerarse, los programas pueden presentar fallos
    importantes, por ejemplo, el famoso videojuego Pacman utilizaba 8 bits para
    el número del nivel en el que se encuentra el jugador y al llegar al nivel
    255 se presentaban una serie de errores que hacían imposible pasar al nivel
    256 (un número binario con 9 bits).
\end{solution}
\end{questions}
\end{document}
