\documentclass[answers]{exam}

\usepackage{amsmath}
\usepackage[spanish]{babel}
\usepackage[T1]{fontenc}
\usepackage{lmodern}
\usepackage{minted}
\usepackage{graphicx}
\usepackage{wrapfig}

\extrawidth{1.04cm}
\extraheadheight[2.5cm]{-5mm}
\renewcommand{\familydefault}{\sfdefault}
\graphicspath{{../../Imagenes/}}

\newcommand{\materia}{Organización y \\ Arquitectura de Computadoras}
\newcommand{\tarea}{Práctica 4}
\newcommand{\titulo}{Lógica Secuencial}
\newcommand{\fecha}{28 de Octubre de 2021}

\firstpageheader{
  \setlength{\intextsep}{2.2em}
  \begin{wrapfigure}{l}{3.3cm}
    \centering
    \includegraphics[scale=0.08]{fc}
  \end{wrapfigure}
  \hfill\break{} \\[3.5mm]
  \LARGE\textbf{\materia} \\
  \large\textbf{Facultad de Ciencias, UNAM} \\[4pt]
  \textbf{José Ethan Ortega González: }316088327 \\
  \textbf{Etzael Iván Sosa Hedding: }316259305 \\
}{}{
  \setlength{\intextsep}{-9.8em}
  \begin{wrapfigure}{l}{3.3cm}
    \centering
    \includegraphics[scale=0.13]{unam}
  \end{wrapfigure}
}

\renewcommand{\solutiontitle}{\noindent\textbf{Solución:}\par\noindent}
\renewcommand{\thequestion}{\textbf{\arabic{question}}}
\runningheadrule{}
\runningfootrule{}
\firstpagefootrule{}
\runningheader{\materia}{\tarea: \titulo}{\fecha}
\footer{}{Página \thepage\ de \numpages}{}

\begin{document}
\begin{questions}
  \question{Un registro de desplazamiento es un circuito secuencial que desplaza
    a la izquierda o a la derecha la información contenida en el. Considerando
    el desplazamiento de $1$ bit a la izquierda, ¿cómo se implementa dicho
    circuito? ¿Cómo podríamos simular su funcionamiento con las operaciones
    que se tienen en la ALU de $8$ bits?}
  \begin{solution}
    Para implementarlo es muy sencillo, solo tenemos que conectar $n$ flip-flops
    de tipo D, donde la entrada del primero puede elegirse y su salida es la
    entrada del siguiente flip-flop, y el reloj está conectado a todos en
    paralelo. Es más conocida como SIPO (Serial Input Parallel Output). Para
    poder simular el funcionamiento con nuestra ALU, simplemente necesitamos
    sumar y restar mediante el siguiente algoritmo:

    \begin{quote}
      Sean $\{n,\ldots,1\}$ los indices de los dígitos en el registro. Buscamos el
      $1$ con mayor indice $k$ y a todo el número le sumamos otro número cuyo
      único $1$ está en la posición $k + 1$, y luego restamos un número cuyo
      único número está en la posición $k$. Repetimos el algoritmo desde $k - 1$
      hasta $1$. Esto nos da un desplazamiento a la izquierda introduciendo un
      $0$, si queremos introducir un $1$ hacemos exactamente lo mismo y le
      sumamos un $1$ al final.
    \end{quote}
  \end{solution}

  \question{Considerando el formato de instrucción para nuestro circuito
    completo (ALU y Banco de Registros conectados) que se muestra en la Tabla
    $1$ escribe las instrucciones necesarias para calcular las siguientes
    operaciones (Suponga que A, B y C se encuentran en los registros 1, 2 y 3
    respectivamente):}
  \begin{itemize}
    \item $2A + B - C$
    \item $(-A \& B) \vert C$
  \end{itemize}
  \begin{solution}
    \begin{itemize}
      \item Las instrucciones son las siguientes:
      \begin{center}
        \begin{verbatim}
          01100000 00000000 // Sumamos dos veces a A y lo guardamos en A
          01100100 01000000 // Sumamos a A y B y lo guardamos en B
          11101001 10000000 // Restamos a B y C y lo guardamos en C
        \end{verbatim}
      \end{center}
      \item Las instrucciones son las siguientes:
      \begin{center}
        \begin{verbatim}
          01000000 00000000 // Negamos las entradas de A y lo guardamos en A
          00000100 01000000 // Hacemos un AND de bits con A y B y lo guardamos en B
          00101001 10000000 // Hacemos un OR de bits con B y C y lo guardamos en C
        \end{verbatim}
      \end{center}
    \end{itemize}
  \end{solution}
\end{questions}
\end{document}
