\documentclass[answers]{exam}

\usepackage{amsmath}
\usepackage[spanish]{babel}
\usepackage{graphicx}
\usepackage{hyperref}
\usepackage{lmodern}
\usepackage{wrapfig}

\decimalpoint{}

\extrawidth{1.04cm}
\extraheadheight[2.5cm]{-5mm}
\renewcommand{\familydefault}{\sfdefault}
\graphicspath{{./Imagenes/}}

\newcommand{\materia}{Organización y Arquitectura \\ de Computadoras}
\newcommand{\tarea}{Tarea 2}
\newcommand{\titulo}{}
\newcommand{\fecha}{28 de Octubre de 2021}

\firstpageheader{
  \setlength{\intextsep}{2.2em}
  \begin{wrapfigure}{l}{3.3cm}
    \centering
    \includegraphics[scale=0.08]{fc}
  \end{wrapfigure}
  \hfill\break{} \\[3.5mm]
  \LARGE\textbf{\materia: \tarea} \\
  \large\textbf{Facultad de Ciencias, UNAM} \\[4pt]
  \textbf{José Ethan Ortega González: }316088327 \\
  \textbf{Etzael Iván Sosa Hedding: }316259305 \\
}{}{
  \setlength{\intextsep}{-9.8em}
  \begin{wrapfigure}{l}{3.3cm}
    \centering
    \includegraphics[scale=0.13]{unam}
  \end{wrapfigure}
}

\renewcommand{\solutiontitle}{\noindent\textbf{Solución:}\par\noindent}
\renewcommand{\thequestion}{\textbf{\arabic{question}}}
\runningheadrule{}
\runningfootrule{}
\firstpagefootrule{}
\runningheader{\materia}{\tarea}{\fecha}
\footer{}{Página \thepage\ de \numpages}{}

\hypersetup{colorlinks=true,urlcolor=blue}

\begin{document}
\thispagestyle{headandfoot}
\begin{questions}
  \question{La Arquitectura de Computadoras se dedica unicamente al estudio de
    las instrucciones de una computadora y su desempeño respecto a estas ¿si,
    no? Argumenta tu respuesta.}
  \begin{solution}
    Siendo estrictos, no. La arquitectura de computadoras se encarga de entender
    el funcionamiento de las computadoras y todo lo que esto involucra:
    \begin{itemize}
      \item Lo que llamamos arquitectura de la computadora no es otra cosa sino
            el cómo se encuentran distribuidos y relacionados los componentes de
            la máquina y qué ventajas o desventajas nos aporta tal diseño, como
            ya vimos con la arquitectura de Von Neuman o de Harvard
      \item Por otra parte, se estudia el hardware de la computadora (la
            memoria, el CPU, el cómo están relacionados los dispositivos
            internos y externos) y su relación con el desempeño de la misma (la
            memoria que tiene y cuánta información se puede almacenar, el tiempo
            que se requiere para realizar ciertas operaciones, las limitaciones
            que tiene, entre otras).
      \item Tampoco podemos dejar de lado el evidente interés por el estudio
            histórico de la computación pues resulta relevante para entender las
            posibilidades del desarrollo tecnológico, las limitantes actuales y
            cómo podemos mejorar las condiciones actuales.
      \item Por último sí resulta vital comprender el desempeño de una
            computadora pues en la mayoría de los casos buscamos optimizar en
            tiempo y espacio todo lo computable, es decir, queremos minimizar el
            gasto de recursos para cada tarea, por más pequeña que esta sea y de
            esta forma encontrar mejores y más seguras soluciones a los
            problemas que se presentan.
    \end{itemize}
  \end{solution}

  \question{¿Los registros son dispositivos de hardware que permiten almacenar
    cualquier valor en binario? Argumenta tu respuesta.}
  \begin{solution}
    Cualquier valor binario siempre y cuando no exceda la capacidad del
    registro. Existen registros con diferentes capacidades (en bits) y pueden
    contener datos en lenguaje binario entre 4 y 64 bits. Los registros se
    encuentran dentro de los procesadores y generalmente almacenan información
    muy utilizada que es necesario consultar de forma rápida.
  \end{solution}


  \question{De los dos tipos de arquitecturas, RISC y CISC.\@ ¿Cuál de las dos
    requiere un mayor numero de instrucciones para realizar una tarea? ¿Por qué
    crees que así sea?}
  \begin{solution}
    La arquitectura RISC requiere más instrucciones. Si bien RISC son las siglas
    para Reduced Instruction Set Computing, la naturaleza misma de la
    arquitectura intenta simplificar las instrucciones, por lo que tareas
    complejas se pueden convertir en muchas tareas más simples (aunque sólo se
    necesita un ciclo de reloj para ejecutar cada instrucción). Por su parte,
    CISC que son las siglas para Complex Instruction Set Computing es una
    arquitectura que no se preocupa si tiene instrucciones muy complejas aunque
    puedan tardar más tiempo.
  \end{solution}

  \question{¿En una arquitectura CISC el periodo de una señal de reloj puede ser
    más grande que en una arquitectura RISC?}
  \begin{solution}
    Es cierto que las instrucciones de una arquitectura CISC pueden tomar más de
    un ciclo de reloj porque deja de interesarse por los ciclos de una
    instrucción para hacer un código reducido.
  \end{solution}

  \question{Un programa tiene $1 \times 10^{8}$ instrucciones y se ejecuta en un procesador
    a 5GHz, el 40\% de las instrucciones del programa tarda 2 ciclos de reloj, el
    40\% tarda 4 y el 20\% restante 5 ciclos. ¿Cuánto tiempo tarda el programa?}
  \begin{solution}
    Primero calculamos el $40\%$ y el $20\%$ de las $1 \times 10^{8}$ instrucciones:
    \begin{align*}
      1 \times 10^{8} \cdot 0.4 &= 40 \times 10^{6} \\
      1 \times 10^{8} \cdot 0.2 &= 20 \times 10^{6}
    \end{align*}
    Después utilizamos la ecuación fundamental de desempeño con cada uno de los
    ciclos e instrucciones:
    \begin{align*}
      T_{p} &= \left(\frac{(40 \times 10^{6}) \times 2}{5 \times 10^{9}}\right)
              + \left(\frac{(40 \times 10^{6}) \times 4}{5 \times 10^{9}}\right)
              + \left(\frac{(20 \times 10^{6}) \times 5}{5 \times 10^{9}}\right) \\
            &= 0.016 + 0.032 + 0.02 = 0.068
    \end{align*}
    Por lo que el tiempo del programa es $0.068$ segundos o $69$ mili-segundos.
  \end{solution}
\end{questions}

\begin{thebibliography}{99}
  \bibitem{ejercicio1}{
    \url{https://www.iteshu.edu.mx/reticulas_plan_2010/isc/arquitectura_de_computadoras.pdf} \\
    \url{https://es.wikipedia.org/wiki/Arquitectura_de_computadoras}}
  \bibitem{ejercicio2}{
    \url{http://cv.uoc.edu/annotation/8255a8c320f60c2bfd6c9f2ce11b2e7f/619469/PID_00218272/PID_00218272.html}}
  \bibitem{ejercicio3}{
    \url{https://triton.astroscu.unam.mx/fruiz/introduccion/introduccion_computacion/Arquitectura%20RISC%20vs%20CISC.pdf}}
  \bibitem{ejercicio4}{
    \url{https://www.profesionalreview.com/2021/07/18/risc-vs-cisc/} \\
    \url{https://hardzone.es/tutoriales/rendimiento/x86-vs-arm-consumo-energetico/}
  }
\end{thebibliography}
\end{document}
