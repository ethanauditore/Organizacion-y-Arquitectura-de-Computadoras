\documentclass[answers]{exam}

\usepackage{amsmath}
\usepackage[spanish]{babel}
\usepackage[T1]{fontenc}
\usepackage{graphicx}
\usepackage{karnaugh-map}
\usepackage{kvmap}
\usepackage{lmodern}
\usepackage{wrapfig}

\decimalpoint{}

\extrawidth{1.04cm}
\extraheadheight[3cm]{-5mm}
\renewcommand{\familydefault}{\sfdefault}
\graphicspath{{../../Imagenes/}}

\newcommand{\materia}{Organización y \\ Arquitectura de Computadoras}
\newcommand{\tarea}{Tarea 3}
\newcommand{\titulo}{Lógica Digital}
\newcommand{\fecha}{28 de Octubre de 2021}

\firstpageheader{
  \setlength{\intextsep}{2.2em}
  \begin{wrapfigure}{l}{3.7cm}
    \centering
    \includegraphics[scale=0.09]{fc}
  \end{wrapfigure}
  \hfill\break{} \\[3.5mm]
  \LARGE\textbf{\materia} \\
  \LARGE\textbf{\tarea: \titulo} \\[4pt]
  \large\textbf{Facultad de Ciencias, UNAM} \\[4pt]
  \textbf{José Ethan Ortega González:} 316088327 \\
  \textbf{Etzael Iván Sosa Hedding:} 316259305 \\
}{}{
  \setlength{\intextsep}{-12.5em}
  \begin{wrapfigure}{l}{3.3cm}
    \centering
    \includegraphics[scale=0.15]{unam}
  \end{wrapfigure}
}

\renewcommand{\solutiontitle}{\noindent\textbf{Solución:}\par\noindent}
\renewcommand{\thequestion}{\textbf{\arabic{question}}}
\runningheadrule{}
\runningfootrule{}
\firstpagefootrule{}
\runningheader{\materia}{\tarea}{\fecha}
\footer{}{Página \thepage\ de \numpages}{}

\begin{document}
\begin{questions}
  \question{Da la dualidad de $x \cdot y = x + y$ y verifica la igualdad respecto a
    esta.}
  \begin{solution}
    Aquí va la solución.
  \end{solution}

  \question{Demuestra si la siguiente igualdad es valida:
    $x(\overline{x} + y) = xy$.}
  \begin{solution}
    La demostración es la siguiente:
    \begin{align*}
      x(\overline{x} + y) &= (x\overline{x}) + xy && \text{Por Distributividad} \\
                          &= 0 + xy && \text{Por Complemento} \\
                          &= xy && \text{Por Identidad}
    \end{align*}
  \end{solution}

  \question{Demuestra si la siguiente igualdad es valida:
    $(x + y)(\overline{x} + z)(y + z) = (x + y)(\overline{x} + z)$.}
  \begin{solution}
    \begin{align*}
      (x + y)(\overline{x} + z)(y + z)
      &= [(x + y)\overline{x} + (x + y)z](y + z) C \\
      &= [(x\overline{x} + y\overline{x}) + (xz + yz)](y + z) && \text{Por Distributividad} \\
      &= [0 + y\overline{x} + xz + yz](y + z) && \text{Por Complemento} \\
      &= (y\overline{x} + xz + yz)(y + z) && \text{Por Identidad} \\
      &= y\overline{x}(y + z) + xz(y + z) + yz(y + z) && \text{Por Distributividad} \\
      &= y\overline{x}y + y\overline{x}z + xzy + xzz + yzy + yzz && \text{Por Distributividad} \\
      &= yy\overline{x} + y\overline{x}z + xzy + xzz + yyz + yzz && \text{Por Conmutatividad} \\
      &= y\overline{x} + y\overline{x}z + xzy + xz + yz && \text{Por Idempotencia} \\
      &= \overline{x}zy + xzy + y\overline{x} + xz + yz && \text{Por Conmutatividad} \\
      &= (\overline{x} + x)zy + y\overline{x} + xz + yz && \text{Por Distributividad} \\
      &= zy + y\overline{x} + xz + yz && \text{Por Complemento} \\
      &= yz + y\overline{x} + xz + yz && \text{Por Conmutatividad} \\
      &= yz + y\overline{x} + xz && \text{Por Idempotencia} \\
      &= yz + y\overline{x} + xz + 0 && \text{Por Identidad} \\
      &= yz + y\overline{x} + xz + (x\overline{x}) && \text{Por Complemento} \\
      &= (\overline{x}x) + \overline{x}y + zx + zy && \text{Por Conmutatividad} \\
      &= \overline{x}(x + y) + z(x + y) && \text{Por Distributividad} \\
      &=(x + y)(\overline{x} + z) && \text{Por Distributividad}
    \end{align*}
  \end{solution}

  \question{Demuestra si la siguiente igualdad es valida:
    $\overline{x \cdot y} = \overline{x} \cdot \overline{y}$.}
  \begin{solution}
    No es valida, mostremos un contraejemplo. Cuando $x=0, y=1$, por un lado
    tenemos lo siguiente:
    \begin{gather*}
      \overline{x \cdot y} = \overline{0 \cdot 1} = \overline{0} = 1
    \end{gather*}
    Por el otro, tenemos lo siguiente:
    \begin{gather*}
      \overline{x} \cdot \overline{y} = \overline{0} \cdot \overline{1}
      = 1 \cdot 0 = 0
    \end{gather*}
  \end{solution}

  \question{Verifica la siguiente igualdad usando los postulados de Huntington.
    \[
      F(x,y,z) = x+x(\overline{x}+y)+\overline{x}y = x+y
    \]} \vspace{-2em}
  \begin{solution}
    La solución es la siguiente:
    \begin{align*}
      x+x(\overline{x}+y)+\overline{x}y = x+y
      &= x+x\overline{x}+xy+\overline{x}y = x+y && \text{P4a} \\
      &= x+0+xy+\overline{x}y = x+y && \text{P5b} \\
      &= x+xy+\overline{x}y = x+y && \text{P2a} \\
      &= x+y(x+\overline{x}) = x+y && \text{P4a} \\
      &= x+y \cdot 1 = x+y && \text{P5a} \\
      &= x+y = x+y && \text{P2b}
    \end{align*}
  \end{solution}

  \question{Obtén los maxitérminos y mintérminos de la siguiente función.
    \[
      F(x, y, z) = \overline{x} \cdot \overline{y} \cdot \overline{z} \cdot x +%
      \overline{z} \cdot x + z \cdot x + y \cdot \overline{y} + \overline{z}
    \]} \vspace{-2em}
  \begin{solution}
    \begin{align*}
      \overline{x} \cdot \overline{y} \cdot \overline{z} \cdot x +%
      \overline{z} \cdot x + z \cdot x + y \cdot \overline{y} + \overline{z}
      &= x \cdot \overline{x} \cdot \overline{y} \cdot \overline{z} +%
        \overline{z} \cdot x + z \cdot x + y \cdot \overline{y} + \overline{z}%
      && \text{Por Conmutatividad} \\
      &= 0 \cdot \overline{y} \cdot \overline{z} +%
        \overline{z} \cdot x + z \cdot x + 0 + \overline{z} && \text{Por Complemento} \\
      &= \overline{z} \cdot x + z \cdot x + 0 + \overline{z} && \text{Por Aniquilación} \\
      &= \overline{z} \cdot x + z \cdot x + \overline{z} && \text{Por Identidad} \\
      &=  x \cdot \overline{z} + x \cdot z + \overline{z} && \text{Por Conmutatividad} \\
      &=  x (\overline{z} + z) + \overline{z} && \text{Por Distributividad} \\
      &=  x \cdot 1 + \overline{z} && \text{Por Complemento} \\
      &=  x + \overline{z} && \text{Por Identidad} \\
    \end{align*}
    La tabla de la expresión al reducirla es la siguiente:
    \begin{gather*}
      \begin{array}{|c|c|c|c|}
        \hline
        x & z & \overline{z} & x + \overline{z} \\
        \hline
        0 & 0 & 0 & 0 \\
        0 & 1 & 0 & 0 \\
        0 & 0 & 1 & 1 \\
        0 & 1 & 1 & 1 \\
        1 & 0 & 0 & 1 \\
        1 & 1 & 0 & 1 \\
        1 & 0 & 1 & 1 \\
        1 & 1 & 1 & 1 \\
        \hline
      \end{array}
    \end{gather*}
    \begin{itemize}
      \item \textbf{Minitérminos:} Recordemos que los minitérminos se fijan en
            las filas que se evalúan a 1 y se niegan los 0. Así, tenemos lo
            siguiente:
      \begin{align*}
        \overline{x} \cdot \overline{z} + x \cdot \overline{z} + x \cdot z
        &= \overline{z} \cdot \overline{x} + \overline{z} \cdot x + x \cdot z
        &&\text{Por Conmutatividad}\\
        &= \overline{z}(\overline{x} + x) + xz &&\text{Por Distributividad}\\
        &= \overline{z} \cdot 1 + xz &&\text{Por Complemento}\\
        &= \overline{z} + xz
      \end{align*}
      \item \textbf{Maxitérminos:} Recordemos que los maxitérminos se fijan en
            las filas que se evalúan a 0 y se niegan los 1. Así, tenemos lo
            siguiente:
      \begin{align*}
        x + \overline{z}
      \end{align*}
    \end{itemize}

  \end{solution}

  \question{Simplifica la siguiente función usando su tabla de verdad asociada.
    \[
      F(x, y, z) = \overline{xyz} + \overline{xy}z + \overline{x}y\overline{z}%
      + x\overline{yz} + \overline{x}yz + x\overline{y}z + xyz
    \]} \vspace{-2em}
  \begin{solution}
    La tabla es la siguiente:
    \begin{gather*}
      \begin{array}{|c|c|c|c|c|c|c|c|c|c|c|}
        \hline
        x & y & z & \overline{xyz} & \overline{xy}z & \overline{x}y\overline{z} &
        x\overline{yz} & \overline{x}yz & x\overline{y}z & xyz & F \\
        \hline
        0 & 0 & 0 & 1 & 0 & 0 & 0 & 0 & 0 & 0 & 0 \\
        0 & 0 & 1 & 1 & 1 & 0 & 0 & 0 & 0 & 0 & 0 \\
        0 & 1 & 0 & 1 & 0 & 1 & 0 & 0 & 0 & 0 & 0 \\
        0 & 1 & 1 & 1 & 1 & 0 & 0 & 1 & 0 & 0 & 0 \\
        1 & 0 & 0 & 1 & 0 & 0 & 1 & 0 & 0 & 0 & 0 \\
        1 & 0 & 1 & 1 & 1 & 0 & 1 & 0 & 1 & 0 & 0 \\
        1 & 1 & 0 & 1 & 0 & 0 & 1 & 0 & 0 & 0 & 0 \\
        1 & 1 & 1 & 0 & 0 & 0 & 0 & 0 & 0 & 1 & 0 \\
        \hline
      \end{array}
    \end{gather*}
  \end{solution}

  \question{Expande la siguiente función y da su maxitérminos.
    \[
      F(x, y, z) = (x + \overline{x}z) \cdot (\overline{y} + \overline{z}) \cdot z
    \]} \vspace{-2em}
  \begin{solution}
    Aquí va la solución.
  \end{solution}

  \question{Utilizando Mapas de Karnaugh simplifica la función.
    \[
      F(x_0, x_1, x_2, x_3) = \overline{x_0 x_1 x_2 x_3} + \overline{x_0 x_1%
        x_2}x_3 + \overline{x_0 x_1 }x_2 x_3 + x_0 \overline{x_1 }x_2 x_3 + x_0%
      x_1\overline{ x_2 x_3} + \overline{x_0}x_1\overline{ x_2 x_3}+ x_0 x_1 x_2 x_3
    \]} \vspace{-2em}
  \begin{solution}
    Aquí va la solución.
  \end{solution}

  \question{Para realizar una Mapa de Karnaugh con más de 5 variables se
    menciono que existe más de una forma de representarlo.

    Investiga ambos métodos y utiliza el que más se te acomode para reducir la
    siguiente función.
    \begin{align*}
      F(x_0, x_1, x_2, x_3, x_4)
      =&\;\overline{x_0 x_1 x_2 x_3 x_4} + \overline{x_0 x_1 x_2}x_3 \overline{x_4}
         + \overline{x_0 x_1 }x_2 x_3 \overline{x_4} + x_0 \overline{x_1 }x_2 x_3 x_4 \\
       &+ x_0 x_1\overline{ x_2 x_3} x_4 + \overline{x_0}x_1\overline{ x_2 x_3}x_4
         + x_0 x_1 x_2 x_3 x_4
    \end{align*}} \vspace{-2em}
  \begin{solution}
    \begin{center}
      \begin{kvmap}
        \kvlist{8}{4}
        {0,0,0,0,0,0,0,0,0,0,0,0,0,0,0,0,0,0,0,0,0,0,0,0,0,0,0,0,0,0,0,0,}
        {x_{2},x_{3},x_{4},x_{0},x_{1}}
        \bundle[color=red]{1}{1}{1}{2}
        \bundle{3}{2}{2}{3}
        \bundle[color=blue,reducespace=3pt]{3}{2}{3}{1}
      \end{kvmap}
    \end{center}
  \end{solution}

  \question{A lo largo del capitulo abordamos los postulados de Huntington y sus
    demostraciones, pero existe un principio llamado Principio de Dualidad, el
    cual nos permite formalizar que a toda relación o ley lógica le
    corresponderá su dual, formada mediante el intercambio de los operadores
    unión (suma lógica) con los de intersección (producto lógico), y de los 1
    con los 0.

    Con esta definición indica cual es la expresión dual de:
    \begin{parts}
      \part{$x \cdot 0 = 0$}
      \part{$x(x + y) = x$}
    \end{parts}}
  \begin{solution}
    \begin{itemize}
      \item Por el Principio de Dualidad, la expresión es: $x+1=1$
      \item Por el Principio de Dualidad, la expresión es: $x+(xy)=x$
    \end{itemize}
  \end{solution}
\end{questions}
\end{document}
