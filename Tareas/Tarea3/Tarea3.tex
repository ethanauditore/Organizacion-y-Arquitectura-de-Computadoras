\documentclass[answers]{exam}

\usepackage{amsmath}
\usepackage[spanish]{babel}
\usepackage[T1]{fontenc}
\usepackage{graphicx}
\usepackage{lmodern}
\usepackage{wrapfig}

\decimalpoint{}

\extrawidth{1.04cm}
\extraheadheight[3.5cm]{-5mm}
\renewcommand{\familydefault}{\sfdefault}
\graphicspath{{../../Imagenes/}}

\newcommand{\materia}{Organización y \\ Arquitectura de Computadoras}
\newcommand{\tarea}{Tarea 3}
\newcommand{\titulo}{Lógica Digital}
\newcommand{\fecha}{28 de Octubre de 2021}

\firstpageheader{
  \setlength{\intextsep}{2.2em}
  \begin{wrapfigure}{l}{3.7cm}
    \centering
    \includegraphics[scale=0.09]{fc}
  \end{wrapfigure}
  \hfill\break{} \\[3.5mm]
  \LARGE\textbf{\materia} \\
  \LARGE\textbf{\tarea: \titulo} \\[4pt]
  \large\textbf{Facultad de Ciencias, UNAM} \\[4pt]
  \textbf{José Ethan Ortega González:} 316088327 \\
  \textbf{Etzael Iván Sosa Hedding:} 316259305 \\
}{}{
  \setlength{\intextsep}{-12.5em}
  \begin{wrapfigure}{l}{3.3cm}
    \centering
    \includegraphics[scale=0.15]{unam}
  \end{wrapfigure}
}

\renewcommand{\solutiontitle}{\noindent\textbf{Solución:}\par\noindent}
\renewcommand{\thequestion}{\textbf{\arabic{question}}}
\runningheadrule{}
\runningfootrule{}
\firstpagefootrule{}
\runningheader{\materia}{\tarea}{\fecha}
\footer{}{Página \thepage\ de \numpages}{}

\begin{document}
\begin{questions}
  \question{Da la dualidad de $x \cdot y = x + y$ y verifica la igualdad respecto a
    esta.}
  \begin{solution}
    Aquí va la solución.
  \end{solution}

  \question{Demuestra si la siguiente igualdad es valida:
    $x(\overline{x} + y) = xy$.}
  \begin{solution}
    Aquí va la solución.
  \end{solution}

  \question{Demuestra si la siguiente igualdad es valida:
    $(x + y)(\overline{x} + z)(y + z) = (x + y)(\overline{x} + z)$.}
  \begin{solution}
    Aquí va la solución.
  \end{solution}

  \question{Demuestra si la siguiente igualdad es valida:
    $\overline{x \cdot y} = \overline{x} \cdot \overline{y}$.}
  \begin{solution}
    Aquí va la solución.
  \end{solution}

  \question{Verifica la siguiente igualdad usando los postulados de Huntington.
    \[
      F(x,y,z) = x + x(\overline{x}+y)+\overline{x}y= x+y
    \]} \vspace{-2em}
  \begin{solution}
    Aquí va la solución.
  \end{solution}

  \question{Obtén los maxitérminos y mintérminos de la siguiente función.
    \[
      F(x, y, z) = \overline{x} \cdot \overline{y} \cdot \overline{z} \cdot x +%
      \overline{z} \cdot x + z \cdot x + y \cdot \overline{y} + \overline{z}
    \]} \vspace{-2em}
  \begin{solution}
    Aquí va la solución.
  \end{solution}

  \question{Simplifica la siguiente función usando su tabla de verdad asociada.
    \[
      F(x, y, z) = \overline{xyz} + \overline{xy}z + \overline{x}y\overline{z}%
      + x\overline{yz} + \overline{x}yz + x\overline{y}z + xyz
    \]} \vspace{-2em}
  \begin{solution}
    Aquí va la solución.
  \end{solution}

  \question{Expande la siguiente función y da su maxitérminos.
    \[
      F(x, y, z) = (x + \overline{x}z) \cdot (\overline{y} + \overline{z}) \cdot z
    \]} \vspace{-2em}
  \begin{solution}
    Aquí va la solución.
  \end{solution}

  \question{Utilizando Mapas de Karnaugh simplifica la función.
    \[
      F(x_0, x_1, x_2, x_3) = \overline{x_0 x_1 x_2 x_3} + \overline{x_0 x_1%
        x_2}x_3 + \overline{x_0 x_1 }x_2 x_3 + x_0 \overline{x_1 }x_2 x_3 + x_0%
      x_1\overline{ x_2 x_3} + \overline{x_0}x_1\overline{ x_2 x_3}+ x_0 x_1 x_2 x_3
    \]} \vspace{-2em}
  \begin{solution}
    Aquí va la solución.
  \end{solution}

  \question{Para realizar una Mapa de Karnaugh con más de 5 variables se
    menciono que existe más de una forma de representarlo.

    Investiga ambos métodos y utiliza el que más se te acomode para reducir la
    siguiente función.
    \begin{align*}
      F(x_0, x_1, x_2, x_3, x_4)
      =&\;\overline{x_0 x_1 x_2 x_3 x_4} + \overline{x_0 x_1 x_2}x_3 \overline{x_4}
         + \overline{x_0 x_1 }x_2 x_3 \overline{x_4} + x_0 \overline{x_1 }x_2 x_3 x_4 \\
       &+ x_0 x_1\overline{ x_2 x_3} x_4 + \overline{x_0}x_1\overline{ x_2 x_3}x_4
         + x_0 x_1 x_2 x_3 x_4
    \end{align*}} \vspace{-2em}
  \begin{solution}
    Aquí va la solución.
  \end{solution}

  \question{A lo largo del capitulo abordamos los postulados de Huntington y sus
    demostraciones, pero existe un principio llamado Principio de Dualidad, el
    cual nos permite formalizar que a toda relación o ley lógica le
    corresponderá su dual, formada mediante el intercambio de los operadores
    unión (suma lógica) con los de intersección (producto lógico), y de los 1
    con los 0.

    Con esta definición indica cual es la expresión dual de:
    \begin{parts}
      \part{$x \cdot 0 = 0$}
      \part{$x(x + y) = x$}
    \end{parts}}
  \begin{solution}
    Aquí va la solución.
  \end{solution}
\end{questions}
\end{document}
